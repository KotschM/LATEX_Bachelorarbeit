
~

\vspace{17.1mm}

\begin{flushleft}
    \textbf{\huge{}Zusammenfassung}{\huge\par}
\par\end{flushleft}

Durch den stetigen Wachstum moderner Softwaresysteme wächst auch der Anteil von ungenutzten Code-Passagen und Features. Innerhalb dieser Praxisarbeit soll ein erster Einblick in die Analysemöglichkeiten für die Identifikation von ungenutztem Code aufgezeigt werden. Um solch eine Analyse durchzuführen, bedient man sich den Ergebnissen der beiden Analyse-Arten \glqq statische Codeanalyse\grqq{} und \glqq dynamische Codeanalyse\grqq{}. Neben einer Erläuterung beider Analyse-Arten wird im Weiteren ein Konzept und die dazugehörige Umsetzung für die dynamische Codeanalyse, anhand des Enisco Manufacturing Execution System (kurz E-MES), der Firma Enisco by Forcam GmbH, aufgezeigt.
