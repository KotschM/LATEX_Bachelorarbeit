\chapter{Einleitung\label{chap1:Erstes-Kapitel}}

\section{Aufgabenstellung\label{sec1.2:Unterpunkt-2}}

Viele historisch gewachsene Systeme sammeln über die Jahre Code an, der ungenutzt ist oder im Laufe der Entwicklung ungenutzt wird. Gründe hierfür können zum einen sein, dass sich Anforderungen und damit der Code geändert haben, so dass Bereiche gar nicht mehr aufgerufen werden können. Zum Anderen kann es sich um Features handeln, die vom Benutzer nicht entdeckt und daher nie genutzt wurden. Da meist unbekannt ist, welcher Code nutzlos ist, verursacht er oft Kosten ohne Wert zu stiften. Bei großen grundlegenden Änderungen an der Software erhöht sich beispielsweise der Wartungsaufwand, da der unnütze Code weiterhin enthalten ist. Deshalb ist wünschenswert ungenutzten Code zu erkennen, um Aufwände einzusparen oder auch die Usability für den Benutzer zu verbessern, indem ungenutzte Features entfernt werden. In der Praxisarbeit sollen verschiedene Methoden und Werkzeuge untersucht werden, mit welchen man ungenutzte Codestellen finden kann. Ziel der Praxisarbeit ist die Entwicklung eines Konzeptes für die Einführung eines Verfahrens zu einer solchen Codeanalyse für das Enisco Manufacturing Execution System (kurz E-MES) und einer anschließenden möglichen Umsetzung.

\section{Vorgehensweise\label{sec1.3:Unterpunkt-3}}

Zunächst soll in \autoref{chap2:Zweites-Kapitel} (\nameref{chap2:Zweites-Kapitel}) eine Einarbeitung in die Methoden zur Identifizierung von ungenutztem Code erfolgen. Dazu werden die verschiedenen Analysemöglichkeiten näher erläutert und gegenüber gestellt. Folgend an diese Einarbeitung soll anschließend in \autoref{chap3:Drittes-Kapitel} (\nameref{chap3:Drittes-Kapitel}) ein Konzept zur Implementierung der ausgewählten Analysemöglichkeiten erarbeitet werden. Ebenso in  \autoref{chap3:Drittes-Kapitel} (\nameref{chap3:Drittes-Kapitel}) wird eine mögliche Umsetzung des Konzeptes vorgestellt.

Abschließend werden in \autoref{chap4:Viertes-Kapitel} (\nameref{chap4:Viertes-Kapitel}) Anregungen für aufbauende Arbeiten gegeben.