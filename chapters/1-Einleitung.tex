\chapter{Einleitung\label{chap1:Erstes-Kapitel}}

Im Rahmen der vorliegenden Arbeit werden verschiedene Umsetzungsmöglichkeiten für eine Suchfunktionalität in einem MES-Produkt der Firma Enisco betrachtet. Der Fokus liegt hierbei auf den diversen Arten für die Umsetzung einer Suche. Durch die architekturelle Neugestaltung des bisherigen MES-Produktes wird auch auf die technischen Besonderheiten bezüglich der Suchfunktionalität in einer Microservice-Architektur eingegangen.

Im folgenden, einleitenden Kapitel wird die betreuende Firma mit dem dazugehörigem Kernprodukt E-MES, die Aufgabenstellung und das geplante Vorgehen erläutert. Um das Lesen der Arbeit zu erleichtern, wird ein Überblick über den Aufbau der Arbeit gegeben.

\section{Enisco und E-MES\label{sec1.1:unterpunkt-1}}

Die vorliegende Arbeit wurde im Rahmen einer Bachelorarbeit bei der Firma Enisco verfasst. Enisco GmbH \& Co. KG wurde 2015 als Tochtergesellschaft der Eisenmann SE gegründet und ist mittlerweile ein eigenständiges Unternehmen, welches unter dem Namen \glqq Enisco by Forcam GmbH\footnote{Im Folgenden wird aus Gründen der Lesbarkeit auf die Rechtsform der Enisco by Forcam GmbH verzichtet}\grqq{} agiert.

Als Kernprodukt vertreibt die Firma Enisco das Produktionsleitsystem \glqq \gls{emes}\grqq{}. Dieses wird für die Überwachung und Steuerung von Produktionsanlagen eingesetzt. E-MES vernetzt dabei die Anlage sowohl horizontal, über den gesamten Fertigungsprozess, als auch vertikal, über alle Prozessebenen hinweg und bildet so ein System, welches zwischen Unternehmensebene (ERP, engl. für Enterprise Ressource Planning) und Steuerungsebene (PLC, engl. für Programmable Logic
Controller) agiert. \cite{EniscobyForcamGmbH.2021b}

E-MES ist dafür modular aufgebaut und besitzt als Basis ein Platform-Modul. Dieses beinhaltet alle Grundlagen und Schnittstellen für die Installation weiterer Module. Die zusätzlichen Module (engl. Add-Ons) ergänzen E-MES um bestimmte Funktionen und können kundenspezifisch installiert und konfiguriert werden.

\section{Aufgabenstellung\label{sec1.2:Unterpunkt-2}}

Derzeit erfährt das aktuelle Produktionsleitsystem E-MES eine Neugestaltung. Dabei wird von einer monolithisch betriebenen, modularen 3-Schichten-Architektur auf eine verteilte Microservice-Architektur gewechselt. Ein Wechsel der Architektur beruht auf dem Eintritt der Muttergesellschafft Forcam GmbH in die \glqq Open Industry 4.0 Alliance\grqq{}. Durch den Zusammenschluss von mehreren Unternehmen aus dem Bereich \glqq Industrie 4.0\grqq{} können einheitliche Schnittstellen definiert werden, um so die Interoperabilität zwischen den Softwarelösungen der beteiligten Firmen zu stärken \cite{OpenIndustry4.0Alliance.2021}. Um die benötigte Interoperabilität zu ermöglichen, setzen die beteiligten Unternehmen vermehrt auf Technologien wie Docker und Kubernetes. Um nun auch die Neugestaltung von E-MES in diesem Umfeld anzubieten, wurde sich für eine verteilte Microservice-Architektur entschieden. Neben der Architektur wird auch der Produktname von \glqq E-MES\grqq{} in den vorläufigen Produktnamen \glqq \gls{mcc}\grqq{} abgeändert.

% Im Zuge der Neugestaltung von E-MES werden neue Funktionalitäten, wie eine Suchfunktion integriert. Eine Suchfunktion in modernen Informationssystemen wird von den Benutzern als gewohnter Komfort wahrgenommen. Auftretende Problemstellungen können dabei mithilfe von Suchabfragen gelöst werden. Solch eine Funktion liefert dem Benutzer eine Liste mit Suchergebnissen, aus der er den geeignetsten Treffer auswählen kann.

Im Zuge der Neugestaltung von E-MES werden neue Funktionalitäten, wie eine Suchfunktion integriert. Eine Suchfunktion in modernen Informationssystemen wird von den Benutzern als gewohnter Komfort wahrgenommen. Für die Umsetzung einer Suchfunktion in einem \gls{mes} muss festgelegt werden, welche Funktionen und Inhalte des \gls{mes} von der Suchfunktion abgedeckt werden sollen. Neben der Suche nach Funktionalitäten des Systems, kann es auch hilfreich sein, nach bestimmten \glqq Objekten\grqq{} innerhalb des Systems zu suchen. Solche Objekte können zum Beispiel in Form von eindeutigen Aufträgen oder Maschinen in einem \gls{mes} vorkommen. Gibt der Benutzer die Kennung eines Objektes in das Suchfeld ein, sollen ihm alle Funktionen und Informationen bezüglich dieses Objektes angezeigt werden. So soll der Benutzer bei der Navigation durch das \gls{mes} unterstützt werden.

% Für die Umsetzung einer Suchfunktion in einem \gls{mes} muss festgelegt werden, welche Funktionen und Inhalte des \gls{mes} von der Suchfunktion abgedeckt werden sollen. Neben der Suche nach Funktionalitäten des Systems, kann es auch hilfreich sein, nach bestimmten \glqq Objekten\grqq{} innerhalb des Systems zu suchen. Solche Objekte können zum Beispiel in Form von eindeutigen Aufträgen oder Maschinen in einem \gls{mes} vorkommen. Gibt der Benutzer die Kennung eines Objektes in das Suchfeld ein, sollen ihm alle Funktionen und Informationen bezüglich dieses Objektes angezeigt werden.

Neben der Anforderungsklärung bezüglich der Suchoptionen und der Granularität der Suchanfragen, gilt es auch eine Konzeption für die Integration einer Search Engine in die MCC-Gesamtarchitektur zu entwerfen. Hierbei sind die besonderen Anforderungen zu beachten, welche durch die Einführung der verteilten Microservice-Architektur entstanden sind. So ist zu klären, welche Strategie für die Datenaktualisierung zwischen einer Search Engine und den Datenhaltungsschichten der einzelnen Services den Anforderungen am besten entspricht. Anhand von selbstgewählten technischen und lizenzbezogenen Kriterien sollen diesbezüglich Strategien und auch potentiell geeignete Search Engines verglichen werden. Bei der Konzeption für die Integration einer Search Engine sind monolithische Seiteneffekte, die durch die Missachtung von Prinzipien der Microservice-Architektur entstehen könnten, zu vermeiden.

Das erstellte Konzept gilt es anschließend mithilfe einer prototypischen Implementierung einer Suchanwendung als \glqq Proof of Concept\grqq{} umzusetzen.

\section{Vorgehensweise und Aufbau der Arbeit\label{sec1.3:Unterpunkt-3}}

\textbf{Folgender Text muss nochmals überarbeitet werden!}

Die Vorgehensweise und die schriftliche Ausarbeitung der vorliegenden Arbeit gliedert sich in drei Hauptteile. Als Vorarbeit für die eigentliche Bearbeitung werden in \textbf{\autoref{chap2:Zweites-Kapitel}} die theoretischen Grundlagen über die verteilte Microservice-Architektur, den Search Engines und dem Entwicklungsstand von \gls{mcc} beschrieben.

Im ersten Schritt wird definiert, mit welchem Suchumfang die Search Engine innerhalb von \gls{mcc} nach Funktionen und Objekten agieren soll. Da zum Zeitpunkt der Erstellung dieser Arbeit noch keine produktreife Version von \gls{mcc} existiert, wird sich an dem Produktumfang und den Funktionalitäten des aktuellen Produktes \gls{emes} orientiert. Im \textbf{\autoref{chap3:Drittes-Kapitel}} wird der Suchumfang für die Suchfunktionalität definiert. Hierbei wird bei \gls{emes} analysiert, welche Objekte innerhalb des Systems \glqq suchbar\grqq{} gemacht werden sollen.

Ein weiterer Schritt ist die Konzeption für die Integration einer Search Engine in \gls{mcc}. Es werden hierbei in \textbf{\autoref{chap4:Viertes-Kapitel}} verschiedene Search Engines anhand von technischen und lizenzbezogenen Kriterien miteinander verglichen. Ebenso werden verschiedene Möglichkeiten der Datenaktualisierung zwischen einer Search Engine und den Datenhaltungsschichten der einzelnen Services erläutert und anhand der erarbeiteten Entscheidungskriterien miteinander verglichen. Vorbereitend für die prototypische Umsetzung wird zusätzlich ein Gesamtkonzept erstellt.

Anschließend an die Konzeption folgt in \textbf{\autoref{chap5:Fuenftes-Kapitel}} eine Beschreibung der prototypische Umsetzung anhand einer Proof-of-Concept-Anwendung. Hierfür wird zunächst festgelegt, welchen Umfang jene prototypische Umsetzung besitzen soll und ob bereits Softwareteile aus \gls{mcc} verwendet werden können.

Abgeschlossen wird die Arbeit mit einem Fazit und einem Ausblick \textbf{(\autoref{chap6:Sechstes-Kapitel})}.